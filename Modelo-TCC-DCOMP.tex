% ------------------------------------------------------------------------
% ------------------------------------------------------------------------
% abnTeX2: Modelo de Trabalho Acadêmico (tese de doutorado, dissertação de
% mestrado e trabalhos monográficos em geral) em conformidade com 
% as normas da ABNT
% ------------------------------------------------------------------------
% ------------------------------------------------------------------------

\documentclass{dcomp-abntex2}

% Geração de dummy text
% Retirar para a versão final do documento
\usepackage{lipsum}


%Compila o indice
\makeindex

\begin{document}

% Seleciona o idioma do documento (conforme pacotes do babel)
\selectlanguage{brazil}

% Retira espaço extra obsoleto entre as frases.
\frenchspacing 

% ----------------------------------------------------------
% ELEMENTOS PRÉ-TEXTUAIS
% ----------------------------------------------------------
\pretextual

\titulo{Utilizando Algoritmos Mágicos para Resolver Problemas de Bancos de dados Obscuros em Nuvens cumulonimbus}
\autor{Arnold Schwarzenegger da Silva}
\orientador{Andrew S. Tanenbaum}
\coorientador{Donald Knuth}

\imprimircapa
\imprimirfolhaderosto*

\imprimirfichacatalografica


\imprimirfolhadeaprovacao
    
\begin{dedicatoria}
   \vspace*{\fill}
   \centering
   \noindent
   \textit{ Este trabalho é dedicado às crianças adultas que,\\
   quando pequenas, sonharam em se tornar cientistas.} \vspace*{\fill}
\end{dedicatoria}
% ---
\begin{agradecimentos}
Os agradecimentos principais são direcionados à Gerald Weber, Miguel Frasson,
Leslie H. Watter, Bruno Parente Lima, Flávio de Vasconcellos Corrêa, Otavio Real
Salvador, Renato Machnievscz\footnote{Os nomes dos integrantes do primeiro
projeto abn\TeX\ foram extraídos de
\url{http://codigolivre.org.br/projects/abntex/}} e todos aqueles que
contribuíram para que a produção de trabalhos acadêmicos conforme
as normas ABNT com \LaTeX\ fosse possível.

Agradecimentos especiais são direcionados ao Centro de Pesquisa em Arquitetura
da Informação\footnote{\url{http://www.cpai.unb.br/}} da Universidade de
Brasília (CPAI), ao grupo de usuários
\emph{latex-br}\footnote{\url{http://groups.google.com/group/latex-br}} e aos
novos voluntários do grupo
\emph{\abnTeX}\footnote{\url{http://groups.google.com/group/abntex2} e
\url{http://www.abntex.net.br/}}~que contribuíram e que ainda
contribuirão para a evolução do \abnTeX.

\end{agradecimentos}
% ---
\begin{epigrafe}
    \vspace*{\fill}
	\begin{flushright}
		\textit{``Não vos amoldeis às estruturas deste mundo, \\
		mas transformai-vos pela renovação da mente, \\
		a fim de distinguir qual é a vontade de Deus: \\
		o que é bom, o que Lhe é agradável, o que é perfeito.\\
		(Bíblia Sagrada, Romanos 12, 2)}
	\end{flushright}
\end{epigrafe}
% ---
% resumo em português
\setlength{\absparsep}{18pt} % ajusta o espaçamento dos parágrafos do resumo
\begin{resumo}
 Segundo a \citeonline[3.1-3.2]{NBR6028:2003}, o resumo deve ressaltar o
 objetivo, o método, os resultados e as conclusões do documento. A ordem e a extensão
 destes itens dependem do tipo de resumo (informativo ou indicativo) e do
 tratamento que cada item recebe no documento original. O resumo deve ser
 precedido da referência do documento, com exceção do resumo inserido no
 próprio documento. (\ldots) As palavras-chave devem figurar logo abaixo do
 resumo, antecedidas da expressão Palavras-chave:, separadas entre si por
 ponto e finalizadas também por ponto.

 \textbf{Palavras-chave}: latex. abntex. editoração de texto.
\end{resumo}
% resumo em inglês
\setlength{\absparsep}{18pt} % ajusta o espaçamento dos parágrafos do resumo
\begin{resumo}[Abstract]
% \begin{otherlanguage*}{english}
   This is the english abstract.

   \vspace{\onelineskip}
 
   \noindent 
   \textbf{Keywords}: latex. abntex. text editoration.
% \end{otherlanguage*}
\end{resumo}
    
\pdfbookmark[0]{\listfigurename}{lof}
\listoffigures*
\cleardoublepage
    
\pdfbookmark[0]{\listtablename}{lot}
\listoftables*
\cleardoublepage
    
% ---
% inserir lista de abreviaturas e siglas
% ---

\begin{siglas}
  \item[ABNT] Associação Brasileira de Normas Técnicas
  \item[abnTeX] ABsurdas Normas para TeX
\end{siglas}
% ---
% ---
% inserir lista de símbolos
% ---

\begin{simbolos}
  \item[$ \Gamma $] Letra grega Gama
  \item[$ \Lambda $] Lambda
  \item[$ \zeta $] Letra grega minúscula zeta
  \item[$ \in $] Pertence
\end{simbolos}
% ---
    
\pdfbookmark[0]{\contentsname}{toc}
\tableofcontents*
\cleardoublepage

% ----------------------------------------------------------
% ELEMENTOS TEXTUAIS
% ----------------------------------------------------------
\textual
\chapter{Introdução}

\lipsum


\section{Motivação}
\lipsum


\section{Objetivos do Trabalho}

\lipsum

\subsection{Obgetivos Gerais}
\lipsum

\subsubsection{Objetivos Mais Gerais Ainda}
\lipsum

\section{Estrutura do Documento}

Para facilitar a navegação e melhor entendimento, este documento está
estruturado em capítulos e seções, que são:
\begin{itemize}
\item {Capítulo 1 - Introdução}: xxx \cite{Yu:2004:ESG:1015090.1015207};
\item {Capítulo 2 - Conceitos Básicos}: xxx \cite{Cormen:2009};
\item {Capítulo 3 - Estado da Arte}: xxx \cite{Weicker:1984:DSS:358274.358283}
\item {Capítulo 4 - Trabalho Propost}o: xxx \cite{IEEE_802_11:6178212};
\item {Capítulo 5 - Resultados}: xxx \cite{Linux:402081};
\item {Capítulo 6 - Conclusão}: xxx \cite{SBC:2012};
\end{itemize}
\chapter{Trabalhos Relacionados}

\section{Seção 1}

\lipsum

\section{Seção 2}

\lipsum

\section{Seção 3}

\lipsum

\section{Considerações Finais}

\lipsum
\chapter{Conclusão}

\lipsum

% ----------------------------------------------------------
% ELEMENTOS PÓS-TEXTUAIS
% ----------------------------------------------------------
\postextual
\include{glossario}
\include{apendices}
\include{anexos}

\bibliography{Bibliografia}

\end{document}