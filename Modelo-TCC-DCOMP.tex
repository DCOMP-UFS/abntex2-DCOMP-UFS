% ------------------------------------------------------------------------
% ------------------------------------------------------------------------
% abnTeX2: Modelo de Trabalho Acadêmico (tese de doutorado, dissertação de
% mestrado e trabalhos monográficos em geral) em conformidade com 
% as normas da ABNT
% ------------------------------------------------------------------------
% ------------------------------------------------------------------------

\documentclass{dcomp-abntex2}

% Geração de dummy text
% Retirar para a versão final do documento
\usepackage{lipsum}


%Compila o indice
\makeindex

\begin{document}

% Seleciona o idioma do documento (conforme pacotes do babel)
\selectlanguage{brazil}

% Retira espaço extra obsoleto entre as frases.
\frenchspacing 

% ----------------------------------------------------------
% ELEMENTOS PRÉ-TEXTUAIS
% ----------------------------------------------------------
\pretextual

\titulo{Utilizando Algoritmos Mágicos para Resolver Problemas de Bancos de dados Obscuros em Nuvens Cumulonimbus}
\autor{Arnold Schwarzenegger da Silva}
\orientador{Andrew S. Tanenbaum}
\coorientador{Donald Knuth}

\imprimircapa
\imprimirfolhaderosto*

\imprimirfichacatalografica


\imprimirfolhadeaprovacao
    
\begin{dedicatoria}
   \vspace*{\fill}
   \centering
   \noindent
   \textit{I dedicate this thesis to all my family, friends and \\ 
   professors who gave me the necessary support to get here.} \vspace*{\fill}
\end{dedicatoria}
% ---
\begin{agradecimentos}

\lipsum[1-4]

\end{agradecimentos}
% ---
\begin{epigrafe}[]
    \vspace*{\fill}
	\begin{flushright}
	
		\textit{Este trabalho, além de cultural, filosófico e pedagógico\\
				É também medicinal, preventivo e curativo\\
				Servindo entre outras coisas para pano branco e pano preto\\
				Curuba e ferida braba\\
				Piolho, chulé e caspa\\
				Cravo, espinha e berruga\\
				Panarismo e água na pleura\\
				Só não cura o velho chifre\\
				Por que não mata a raiz\\
				Pois fica ela encravada\\
				No fundo do coração\\
				(Falcão)}
		
	\end{flushright}
\end{epigrafe}
% ---
% resumo em português
\setlength{\absparsep}{18pt} % ajusta o espaçamento dos parágrafos do resumo
\begin{resumo}
 
A prática cotidiana prova que a expansão dos mercados mundiais não pode mais se dissociar do processo de comunicação como um todo. Todas estas questões, devidamente ponderadas, levantam dúvidas sobre se a necessidade de renovação processual ainda não demonstrou convincentemente que vai participar na mudança das diversas correntes de pensamento. As experiências acumuladas demonstram que a adoção de políticas descentralizadoras acarreta um processo de reformulação e modernização das novas proposições. 

Gostaria de enfatizar que a determinação clara de objetivos apresenta tendências no sentido de aprovar a manutenção das condições inegavelmente apropriadas. O cuidado em identificar pontos críticos no aumento do diálogo entre os diferentes setores produtivos talvez venha a ressaltar a relatividade do fluxo de informações. Pensando mais a longo prazo, a complexidade dos estudos efetuados garante a contribuição de um grupo importante na determinação do sistema de participação geral. O que temos que ter sempre em mente é que a revolução dos costumes estende o alcance e a importância das regras de conduta normativas. Por outro lado, a consolidação das estruturas representa uma abertura para a melhoria do orçamento setorial. 

É claro que a mobilidade dos capitais internacionais aponta para a melhoria das diretrizes de desenvolvimento para o futuro. Evidentemente, o desenvolvimento contínuo de distintas formas de atuação exige a precisão e a definição do levantamento das variáveis envolvidas.

Podemos já vislumbrar o modo pelo qual a determinação clara de objetivos representa uma abertura para a melhoria das diversas correntes de pensamento. O cuidado em identificar pontos críticos na necessidade de renovação processual estimula a padronização dos paradigmas corporativos. A prática cotidiana prova que o início da atividade geral de formação de atitudes garante a contribuição de um grupo importante na determinação do retorno esperado a longo prazo. As experiências acumuladas demonstram que a hegemonia do ambiente político apresenta tendências no sentido de aprovar a manutenção das regras de conduta normativas. 

Desta maneira, a revolução dos costumes afeta positivamente a correta previsão das direções preferenciais no sentido do progresso. Por conseguinte, a determinação clara de objetivos auxilia a preparação e a composição dos métodos utilizados na avaliação de resultados. A certificação de metodologias que nos auxiliam a lidar com o julgamento imparcial das eventualidades apresenta tendências no sentido de aprovar a manutenção dos relacionamentos verticais entre as hierarquias. 

 \textbf{Palavras-chave}: Algoritmos, Computação em Nuvem, Banco de Dados, Lero-Lero.
\end{resumo}
% resumo em inglês
\begin{resumo}[Abstract]
 \begin{otherlanguage*}{english}
   This is the english abstract.

   \vspace{\onelineskip}
 
   \noindent 
   \textbf{Keywords}: latex. abntex. text editoration.
 \end{otherlanguage*}
\end{resumo}
    
\pdfbookmark[0]{\listfigurename}{lof}
\listoffigures*
\cleardoublepage
    
\pdfbookmark[0]{\listtablename}{lot}
\listoftables*
\cleardoublepage
    
% Lista de abreviaturas e siglas

\begin{siglas}
	\item[ABNT]{Associação Brasileira de Normas Técnicas}
	\item[abnTeX]{ABsurdas Normas para TeX}
  	\item[DCOMP]{Departamento de Computação}
	\item[UFS]{Universidade Federal de Sergipe}
\end{siglas}
\include{Configuraveis/Simbolos}
    
\pdfbookmark[0]{\contentsname}{toc}
\tableofcontents*
\cleardoublepage

% ----------------------------------------------------------
% ELEMENTOS TEXTUAIS
% ----------------------------------------------------------
\textual
\chapter{Introdução}

\lipsum


\section{Motivação}
\lipsum


\section{Objetivos do Trabalho}

\lipsum

\subsection{Obgetivos Gerais}
\lipsum

\subsubsection{Objetivos Mais Gerais Ainda}
\lipsum

\section{Estrutura do Documento}

Para facilitar a navegação e melhor entendimento, este documento está
estruturado em capítulos e seções, que são:
\begin{itemize}
\item {Capítulo 1 - Introdução}: xxx \cite{Yu:2004:ESG:1015090.1015207};
\item {Capítulo 2 - Conceitos Básicos}: xxx \cite{Cormen:2009};
\item {Capítulo 3 - Estado da Arte}: xxx \cite{Weicker:1984:DSS:358274.358283}
\item {Capítulo 4 - Trabalho Propost}o: xxx \cite{IEEE_802_11:6178212};
\item {Capítulo 5 - Resultados}: xxx \cite{Linux:402081};
\item {Capítulo 6 - Conclusão}: xxx \cite{SBC:2012};
\end{itemize}
\chapter{Trabalhos Relacionados}

\lipsum

\section{Seção 1}

\lipsum

\section{Seção 2}

\lipsum

\section{Seção 3}

\lipsum

\section{Considerações Finais}

\lipsum
\chapter{Conclusão}

\lipsum[31-33]

% ----------------------------------------------------------
% ELEMENTOS PÓS-TEXTUAIS
% ----------------------------------------------------------
\postextual
\include{glossario}
\include{apendices}
\include{anexos}

\bibliography{Bibliografia}

\end{document}