\chapter{Customização DCOMP}

\section{Pacote Minted}

Usado para criar a lista de códigos, adicionar sintaxe highlight, enumerar as linhas e colorir o fundo, para dar destaque a implementação.

Sintaxe básica:
\begin{verbatim}
\begin{listing}[!htb]
    \caption{Espaço para o título do código}
    \label{Espaço para o label do código, para ser usado na referência}  
    \begin{minted}{Espaço para a linguagem de programação}
        <CÓDIGO>
    \end{minted}
\end{listing}
\end{verbatim}

\begin{listing}[!htb]
    \caption{Primeiro código C}
    \label{labelJava}
    
    \begin{minted}{c}
    int main() {
        printf("hello world");
        return 0;
    }
    \end{minted}
    
\end{listing}

\begin{listing}[!htb]
    \caption{Primeiro código Python}
    \label{labelPython}
	\begin{minted}{python}
import numpy as np
 
def incmatrix(genl1,genl2):
    m = len(genl1)
    n = len(genl2)
    M = None #to become the incidence matrix
    VT = np.zeros((n*m,1), int)  #dummy variable
 
    #compute the bitwise xor matrix
    M1 = bitxormatrix(genl1)
    M2 = np.triu(bitxormatrix(genl2),1) 
 
    for i in range(m-1):
        for j in range(i+1, m):
            [r,c] = np.where(M2 == M1[i,j])
            for k in range(len(r)):
                VT[(i)*n + r[k]] = 1;
                VT[(i)*n + c[k]] = 1;
                VT[(j)*n + r[k]] = 1;
                VT[(j)*n + c[k]] = 1;
 
                if M is None:
                    M = np.copy(VT)
                else:
                    M = np.concatenate((M, VT), 1)
 
                VT = np.zeros((n*m,1), int)
 
    return M
	\end{minted}
\end{listing}

